\documentclass[11pt]{article}
\usepackage{lingmacros}
\usepackage{tree-dvips}
\usepackage{titlesec}

\usepackage[utf8]{inputenc}
\usepackage[T1]{fontenc}
\usepackage{lmodern}
\usepackage[ngerman]{babel}
\usepackage{amsmath, amssymb, amscd, amsthm, amsfonts}
\usepackage{sectsty}
\usepackage{titling}
\usepackage{glossaries}
\usepackage{listings}
\usepackage{xcolor}
\usepackage{graphicx}
\usepackage{tikz}
\usepackage{tikz-qtree}
\usepackage[ngerman]{babel}
\usepackage{verbatim}

%definition the colors 
\definecolor{olive}{RGB}{128,128,0}
\definecolor{gray}{rgb}{0.5,0.5,0.5}
\definecolor{lightSteelBlue}{RGB}{240,240,240}
\definecolor{crimson}{RGB}{220,20,60}
\definecolor{niceBlue}{RGB}{0,210,210}


%listdefinition for C code
\lstdefinestyle{ColorStyle}{
    backgroundcolor=\color{lightSteelBlue},   
    commentstyle=\color{olive},
    keywordstyle=\color{crimson},
    numberstyle=\tiny\color{gray},
    stringstyle=\color{niceBlue},
    basicstyle=\ttfamily\footnotesize,
    breakatwhitespace=false,         
    breaklines=true,                 
    captionpos=b,                    
    keepspaces=true,                 
    numbers=left,                    
    numbersep=5pt,                  
    showspaces=false,                
    showstringspaces=false,
    showtabs=false, 
    language=C,  
    tabsize=2
}

\newcommand{\lstin}[1]{\lstinline[language=C]{#1}}

\renewcommand{\abstractname}{Summary}


\renewcommand\maketitlehooka{\null\mbox{}\vfill}
\renewcommand\maketitlehookd{\vfill\null}


\oddsidemargin 0pt
\evensidemargin 0pt
\marginparwidth 40pt
\marginparsep 10pt
\topmargin -20pt
\headsep 10pt
\textheight 8.7in
\textwidth 6.65in
\linespread{1.2}

%deckblatt 
\title{\textbf{Multimedia Projekt: OpenGL Phasenportraits komplexer Zahlen}}
\author{Siobhan-Lillian Hönig || Fritz Meitner}
\date{\today}

\makeglossaries
\newglossaryentry{Test}{
    name={test},
    description={test}}



\lstset{style=ColorStyle}


\begin{document}
    
\begin{titlepage}
    \maketitle
\end{titlepage}
\pagebreak

\pagebreak
\tableofcontents
\pagebreak


\glsaddall
\printglossary[title=Special terms, toctitle=List of terms]
\pagebreak

\section{Einleitung}
In der Schule als auch Im Studium, werden viele Funktionen und Zahlen definiert und erklärt. 
Wenn zum Beispiel gefragt wird, was eine Sinus Funktion ist und wie diese aussieht, fällt es leicht, sich diese im Allgemeinen grafisch vorzustellen. 
Wenn gefragt ist was natürliche Zahlen sind, können sich diese auch leicht anhand eines Zahlenstrahles oder in einem Koordinatensystem vorgestellt werden. 
Jedoch wenn die Frage aufkommt, was  Komplexe Zahlen sind und wie sie aufgebaut sind, wird es schon schwerer sich diese bildlich darzustellen.
Dieser Arbeit handelt von der Darstellung Komplexer Zahlen in einem Phasenportrait. Es wird zur Darstellung die offene Grafikbibliothek OpenGL genutzt.  
Im Folgenden werden die Komplexen Zahlen sowie OpenGl erklärt und anschließend die Umsetzung in Phasenportraits. 


\section{Komplexe Zahlen} \label{Com}
\subsection{Allgemein}
Komplexe Zahlen erweitern die möglichen Lösungen der 

\section{OpenGL} \label{GL}
\subsection{Allgemein}

\subsection{Aufbau}

\section{Domain Coloring}

\subsection{Aufbau}
\subsection{Implementierung}

\section{fazit}

\pagebreak

\bibliographystyle{alpha}
\bibliography{references}

\end{document}