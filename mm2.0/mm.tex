\documentclass[11pt]{article}
\usepackage{lingmacros}
\usepackage{tree-dvips}
\usepackage{titlesec}

\usepackage[utf8]{inputenc}
\usepackage[T1]{fontenc}
\usepackage{lmodern}
\usepackage[ngerman]{babel}
\usepackage{amsmath, amssymb, amscd, amsthm, amsfonts}
\usepackage{sectsty}
\usepackage{titling}
\usepackage{glossaries}
\usepackage{listings}
\usepackage{xcolor}
\usepackage{graphicx}
\usepackage{tikz}
\usepackage{tikz-qtree}
\usepackage[ngerman]{babel}
\usepackage{verbatim}

%definition the colors 
\definecolor{olive}{RGB}{128,128,0}
\definecolor{gray}{rgb}{0.5,0.5,0.5}
\definecolor{lightSteelBlue}{RGB}{240,240,240}
\definecolor{crimson}{RGB}{220,20,60}
\definecolor{niceBlue}{RGB}{0,210,210}


%listdefinition for C code
\lstdefinestyle{ColorStyle}{
    backgroundcolor=\color{lightSteelBlue},   
    commentstyle=\color{olive},
    keywordstyle=\color{crimson},
    numberstyle=\tiny\color{gray},
    stringstyle=\color{niceBlue},
    basicstyle=\ttfamily\footnotesize,
    breakatwhitespace=false,         
    breaklines=true,                 
    captionpos=b,                    
    keepspaces=true,                 
    numbers=left,                    
    numbersep=5pt,                  
    showspaces=false,                
    showstringspaces=false,
    showtabs=false, 
    language=C,  
    tabsize=2
}

\newcommand\blueColor[1]{\color{niceBlue}#1}
\newcommand{\lstin}[1]{\lstinline[language=C]{#1}}

\renewcommand\maketitlehooka{\null\mbox{}\vfill}
\renewcommand\maketitlehookd{\vfill\null}


\oddsidemargin 0pt
\evensidemargin 0pt
\marginparwidth 40pt
\marginparsep 10pt
\topmargin -20pt
\headsep 10pt
\textheight 8.7in
\textwidth 6.65in
\linespread{1.2}

%deckblatt 
\title{\textbf{Multimedia Projekt: Komplexe Funktionen}}
\author{Siobhan-Lillian Hönig & Fritz Meitner}
\date{\today}

\lstset{style=ColorStyle}

\begin{document}
    
\begin{titlepage}
\maketitle
\end{titlepage}

\pagebreak
\section{Einleitung}
Heutzutage werden wir alltäglich mit Graphen konfrontiert, sei es die Darstellung von in unseren heutigen zeit Geimpften gegen Covid oder simpel die Population von füchsen in Deutschland. Aber auch Funktionen wie der Logarithmus können im Graphen dargestellt werden.
Meistens werden jedoch die Funktionen "simpel und einfach" dargestellt und wirken leblos. Anhand von komplexen Zahlen wird der allgemeine Zahlenbereich erweitert und der imaginare Teil wird eingeführt. Im kapitel \ref{Com} wird dies noch genauer erklärt. 
Um diese Zahlen genauer darstellen zu können, kann die sogenannte Domain Coloring Methode angewendet werden. Diese ermöglicht es die komplexen Zahlen farblich darzustellen. In der folgenden Seminararbeit wird diese Methode genutzt und mithilfe von OpenGL dargestellt.


\section{Komplexe Zahlen} \label{Com}
Komplexe Zahlen erweitern die möglichen Lösungen der 
\section{OpenGL}

\section{Domain Coloring}

\section{fazit}


\pagebreak
\tableofcontents
\pagebreak



\bibliographystyle{alpha}
\bibliography{references}

\end{document}